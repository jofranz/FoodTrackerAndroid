	\documentclass[
    DIV12,
    cleardouble=plain,
    headings=normal,
    pdftex,
    headexclude,footexclude,
    final
]{scrreprt}


%\usepackage{spreadtab}
\usepackage{xspace}
\usepackage[ngerman]{babel}
\usepackage[utf8]{inputenc}
%\usepackage[T1]{fontenc}
\usepackage[pdftex]{graphicx}
\usepackage[bookmarks]{hyperref}
\usepackage{scrpage2}
\usepackage{longtable}
\usepackage{caption}
\usepackage{pgfplots}
\usepackage{float}
\usepackage{xcolor}
\usepackage{colortbl}
\usepackage{tabularx}
\usepackage{multirow} % tabelle
\usepackage{listings} % code aus datei einbinden
\usepackage{scrhack}
\usepackage{comment}


% word wrap with a red arrow
\lstset{
  basicstyle=\ttfamily,
  columns=fullflexible,
  frame=single,
  breaklines=true,
  postbreak=\mbox{\textcolor{red}{$\hookrightarrow$}\space},
}

\graphicspath{{./}{./images/}}

% #################################################################

\hyphenation{Cha-otn-gsch-werl}
\setlength\headheight{1.75cm}

\ihead{\small{Hochschule Hof}}
\chead{}
\ohead{\includegraphics[height=0.05\textheight]{fh_logo}}
\pagestyle{scrheadings}


\setcounter{secnumdepth}{5}
\setcounter{tocdepth}{5}
\renewcommand{\arraystretch}{1}

\parskip0.5\baselineskip plus 0.125\baselineskip minus 0.25\baselineskip
\parindent0em

%\automark[section]{chapter}

\titlehead{\begin{center}\includegraphics[width=5cm]{fh_logo}\end{center}}
 \title{
  Entwicklung einer Android Food Tracking App \\[1em]
  in der Programmiersprache Kotlin  
}
\publishers{Vorgelegt bei Prof. Dr. Michael Stepping}

\author{Johannes Franz, Normen Krug}


\date{Wintersemester 2017/2018}


\begin{document}
\maketitle


\pagenumbering{roman}

\tableofcontents

%\listoftables

\newpage
\pagenumbering{arabic}


\chapter{Einleitung}
Ziel ist es eine App zu entwickeln, die einen Rückschluss von sich zugenommener Nahrung auf Symptome zu ermöglichen. Dabei helfen die in der App eingetragenen Datenpunkte und Fotos diese mit einer z.B. allergischen Reaktion zu verknüpfen.
Zielgruppe sind dabei Menschen, die wegen Erkrankungen aus ihren zu sich zugenommenen Speisen und Getränken Rückschlüsse auf ihr Wohlbefinden treffen wollen.


\newpage

\chapter{Motivation}



\newpage

\chapter{Entwurfsphase}



\newpage

\chapter{Umsetzung}

\newpage

\chapter{Aufgetretene Probleme}

\newpage

\chapter{Lessons learned}



\newpage

\chapter{Weitere Arbeiten}
In diesem Projektabschnitt konnten alle gesteckten Ziele erreicht werden. Da solch ein junges Projekt noch viele Möglichkeiten beinhaltet Funktionen zu verbessern und neue Funktionen einzuführen werden hier mögliche Punkte zur Anregung aufgelistet.
\newpage


\listoffigures

\end{document}