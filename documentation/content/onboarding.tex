\chapter{Onboarding}
Patrick Niepel \& Marcel Hagmann \& Carl Philipp Knoblauch

\section{Einleitung}
Eine weitere Aufgabe, die unsere Gruppe übernommen hat, war das Onboarding für die Stundenplan App. Im Onboarding kann der Nutzer gleich zum Start der App seine Einstellungen zu Fakultät, Studiengang, Semester, Vorlesung und Kalendersynchronisation vornehmen. Das Onboarding wird gestartet, wenn noch keine Angaben zu den genannten Einstellungen gemacht wurden.


\section{Umsetzung}
Zuerst überlegten wir uns, welche Informationen die App vom Nutzer benötigt, damit der Dienst im vollen Umfang verwendet werden kann.
Dabei kamen wir auf folgendes Ergebnis:
\begin{enumerate}
\item Fakultät (die die Farbe der App bestimmt)
\item Abfrage der Fakultät
\item Abfrage des Studiengangs
\item Abfrage des Semesters
\item  Abfrage der besuchten Vorlesungen
\item Ob eine Synchronisation mit dem Kalender gewünscht ist
\item Ob Push-Benachrichigungen gewünscht sind
\end{enumerate}

Das gesamte Onboarding befindet sich in einem separaten Storyboard (Onboarding.storyboard), damit die Übersichtlichkeit des aktuellen Projektes weiterhin gewährleistet wird.
Jeder Schritt im Onboarding (Ausnahme: Kalendersynchronisation) kann nur fortgesetzt werden, wenn eine Auswahl getroffen worden ist, die für die weiteren Schritte zwingend notwendig ist.